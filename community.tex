\section{  COMMUNITY INVOLVEMENT   }
\label{Sec:community}
 
LSST has been conceived as a public facility: the database that it will
produce, and the associated object catalogs that are generated from that
database, will be made available to the U.S. and Chilean scientific communities,
to international partners and to the public at large, with no proprietary period. 
We are working with international partners 
to make LSST data products available worldwide. The LSST data management 
system will provide user-friendly tools to access this database and to support
user-initiated queries, run on LSST computers, either at the archive facility 
or at the data access centers. We expect that many, perhaps even the majority,
of LSST discoveries will come from research astronomers with no formal
affiliation to the project, from students, and from interested amateurs, 
intrigued by the accessibility to the Universe that this facility uniquely 
provides. For example, because of its interest in the large public data access 
of the LSST program, Google has joined the LSST team. 

The SDSS provides a good example for how the scientific 
community can be effective in working with large, publicly available
astronomical data sets. The SDSS has published a series of large incremental
data releases via a sophisticated database, roughly once per year, together with 
a paper describing the content of each data release, and extensive on-line 
documentation giving instructions on downloading the catalogs and image data
(see http://www.sdss.org). The overwhelming majority of the almost
6000 refereed papers based 
on SDSS data to date have been written by scientists from outside 
the project, and  include many of the most high-profile results that have come 
from the facility. 

Nevertheless, it is clear that many of the highest priority LSST science
investigations will require organized teams of professionals working together
to optimize science analyses and to assess the importance of systematic 
uncertainties on the derived results. To meet this need, eleven science
collaborations have been established by the project in core science
areas. For example, the LSST Dark Energy Science Collaboration includes
members with interests in the study of dark energy and related topics in 
fundamental physics with LSST data. As of the time of this contribution, there are 
over 700 participants in these collaborations, mostly from LSST member institutions. 
%Periodic calls for applications for membership are issued to the
%professional  community at large\footnote{See
%http://www.noao.edu/lsst/collab\_prop/Scicollab.htm}. 
One may contact the chairs of each science collaboration to learn the
application process to join; all those at US and Chilean institutions,
as well as named individuals from institutions in other countries
which have signed Memoranda of Agreement to contribute to LSST
operations costs are eligible.  
It is anticipated that the LSST science community  will perform its
analyses using  
LSST computational facilities, and the system has been sized accordingly. 

As we design our observing strategies, we are actively seeking and implementing
input by the LSST science community.  The LSST science collaborations
in particular have helped develop the LSST science case and continue
to provide advice on how to optimize their science with choices in
cadence, software,  
and data systems. The Science Collaborations are expected to play a
role  
in the system optimization during the commissioning period. We will continue 
this review when the survey starts, and will maintain our design of flexible 
cadence structure.

The Science Advisory Committee (SAC), chaired by Michael Strauss, provides a formal, and 
two-way, connection to the external science community served by LSST. This committee takes 
responsibility for policy questions facing the project and also deals with technical topics of interest 
to both the science community and the LSST Project. The SAC minutes and notes are available 
publically. Membership of this committee is as follows: N. Brandt
(Penn State), H. Ferguson (STScI), 
C. Hirata (Ohio State), L. Hunter (UC Sant Cruz), J. Kalirai (STScI), B. Jain (UPenn), M. Kasliwal (Carnegie), 
D. Kirkby (UC Irvine), R. Malhotra (U Arizona), R. Mandelbaum
(Carnegie Mellon),
D. Minniti (U Catolica, Chile), R. Mu\~noz (U. Chile), 
M. Strauss (Princeton), L. Walkowicz (Adler), B. Willman (Haverford), and M. Wood-Vasey 
(U Pittsburgh). 

