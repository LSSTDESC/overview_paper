\section{  COMMUNITY INVOLVEMENT   }
\label{Sec:community}
%Edited by M. Strauss, 11/10/17
% and by Tony Tyson 11/13/17
 
LSST has been conceived as a public facility: the database that it will
produce, and the associated object catalogs that are generated from that
database, will be made available with no proprietary period to the
U.S. and Chilean scientific communities, as well as to those
international partners who contribute to operations funding.  As
described in \S~\ref{Sec:impact}, data will also be made available to
the general public for educational and outreach activities.  
%We are working with international partners 
%to make LSST data products available worldwide. 
The LSST data management 
system will provide user-friendly tools to access this database, support
user-initiated queries and data exploration, and carry out scientific analyses on the
data, using LSST computers either at the archive facility 
or at the data access centers. 
%We could say much more here, based on the science platform plans that
%Mario and others have developed... 
We expect that many, perhaps even the majority,
of LSST discoveries will come from research astronomers with no formal
affiliation to the project, from students, and from interested amateurs, 
intrigued by the accessibility to the Universe that this facility uniquely 
provides. 
%For example, because of its interest in the large public data access 
%of the LSST program, Google has joined the LSST team. 
%[MAS: I took this sentence out, because a decade after we first wrote
%it, we can't really point to a single example of how this connection
%has made a real difference in the way LSST is being done.]
% [TT: I agree, but this is in flux.]

The SDSS provides a good example for how the scientific 
community can be effective in working with large, publicly available
astronomical data sets. The SDSS has published a series of large incremental
data releases via a sophisticated database, roughly once per year, together with 
a paper describing the content of each data release, and extensive on-line 
documentation giving instructions on downloading the catalogs and image data
(see http://www.sdss.org). The overwhelming majority of the almost
8000 refereed papers based 
on SDSS data to date have been written by scientists from outside 
the project, and  include many of the most high-profile results that have come 
from the survey. 

Nevertheless, it is clear that many of the highest priority LSST science
investigations will require organized teams of professionals working together
to optimize science analyses and to assess the importance of systematic 
uncertainties on the derived results. To meet this need, a number of science
collaborations have been established in core science
areas. For example, the LSST Dark Energy Science Collaboration includes
members with interests in the study of dark energy and related topics in 
fundamental physics with LSST data. As of the time of this contribution, there are 
over 800 participants in these collaborations. %, mostly from LSST member institutions. 
The science collaborations are listed on the LSST web page, together
with a description of the application process for each one.  
All those at US and Chilean institutions,
as well as named individuals from institutions in other countries
which have signed Memoranda of Agreement to contribute to LSST
operations costs are eligible to apply. As described in
\S\S~\ref{sec:dm} and \ref{Sec:dp}, LSST will make available
substantial computational resources to the
science community to carry out their analyses; 
%It is anticipated that the LSST science community  will perform its
%analyses using  LSST computational facilities, and 
the system has been sized accordingly. 

As we design our observing strategies, we are actively seeking and implementing
input by the LSST science community.  The LSST science collaborations
in particular have helped develop the LSST science case and continue
to provide advice on how to optimize their science with choices in
cadence, software,  
and data systems. A recent example is the publication of a document
entitled ``Science-Driven Optimization of the LSST Observing
Strategy'' (LSST Science Collaborations 2017), a living document that
quantifies the science returns in different areas for different
observing cadence.  The cadence will continue to be refined, with
input from the science collaborations, during the commissioning, and
the observing strategy will be regularly reviewed, with flexibility
built in, during operations.  
%Science Collaborations are expected to play a role  
%in the system optimization during the commissioning period. We will continue 
%this review when the survey starts, and will maintain our design of flexible 
%cadence structure.

The Science Advisory Committee (SAC), chaired by Michael Strauss,
provides a formal, and two-way, connection to the external science
community served by LSST. This committee takes responsibility for
policy questions facing the project and also deals with technical
topics of interest to both the science community and the LSST
Project. The SAC minutes and notes are available
publicly. Current members on this committee are: T. Anguita (Andr\'es
Bello, Chile), R. Bean (Cornell), W.N. Brandt
(Penn State), J. Kalirai (STScI), M. Kasliwal
(Caltech), D. Kirkby (UC Irvine), C. Liu (Staten Island), A. Mainzer
(JPL), R. Malhotra (U Arizona),
N. Padilla (U. Cat\'olica de Chile), J. Simon (Carnegie), A. Slosar
(Brookhaven), M. Strauss (Princeton), L. Walkowicz (Adler),
and R. Wechsler (Stanford). 

