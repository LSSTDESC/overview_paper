\section{  EDUCATIONAL AND SOCIETAL IMPACTS    }
\label{Sec:impact}

The impact and enduring societal significance of LSST will exceed its direct contributions to advances in physics and astronomy.  LSST is uniquely positioned to have high impact with the interested public, planetariums and science centers, and citizen science projects, as well as middle school through university educational programs. LSST will contribute to the national goals of enhancing science literacy and increasing the global competitiveness of the US science and technology workforce. Engaging the public in LSST activities has been part of the project design from the beginning.

The mission of LSST's Education and Public Outreach (EPO) program is to provide worldwide access to a subset of LSST data through accessible and engaging online experiences so anyone can explore the universe and be part of the discovery process. To do this, LSST EPO will facilitate a pathway from entry-level exploration of astronomical imagery and information to more sophisticated interaction with LSST data using tools similar to what professional astronomers use for their work.

A dynamic, immersive web portal will enable members of the public to explore color images of the full LSST sky, examine objects in more detail, view events from the nightly alert stream, learn more about LSST science topics and discoveries, and investigate scientific questions that excite them using real LSST data in online science notebooks. The portal will also link to numerous citizen science projects using LSST data.

LSST data can become a key part of classrooms emphasizing student-centered research in middle school, high school, and undergraduate settings. Using online science notebooks, teachers will be able to bring real LSST telescope data into their classrooms without having to download, install, and maintain software locally. Educational investigations will be designed to support key aspects of the Next-Generation Science Standards (NGSS) in the USA, and goals of the Explora program through CONICYT in Chile. Educators will be supported through professional development that offers training on the online notebook technology and also relevant science content. Science notebooks will also accommodate access to LSST data for lifelong learners and anyone that visits the portal.

Anyone around the world will be able to participate in a variety of citizen science projects that use LSST data.  The EPO Team will work with the Zooniverse to develop the {\it Project Builder} to include tools specifically designed to utilize LSST data, allowing LSST principal investigators to create any number of projects to help them accomplish their science goals. EPO anticipates that the number of citizen science projects in the astronomy field will increase dramatically when LSST is operational, giving a whole new generation of citizen scientists the opportunity to deepen their engagement with astronomy using authentic data from LSST.

LSST EPO will produce and maintain a digital library of multimedia assets including images, video clips, and 3D models.  Multimedia assets will be aligned to standards such as IMERSA Dome Master and Astronomy Visualization Metadata, when applicable, allowing for the most flexibility for adoption by content creators at planetariums and science centers. We will also follow the International Planetarium Society's Data2Dome standard, which will serve as an additional form of asset distribution.

The LSST EPO program will rely on a cloud-based EPO Data Center (EDC) to handle the unique needs of the EPO audiences. These needs include, for example, a fast and smooth browsing experience on mobile devices, and the need to handle inevitable spikes and lulls in visitor traffic and data transfers. As such, the EDC will follow agility best practices popularized by cloud computing, leveraging on-demand computing and auto-scalable architecture. Remaining agile and relevant during the full lifetime of Operations by adjusting to technology trends and education priorities is an important part of the LSST EPO design process.

LSST EPO is committed to engaging with diverse audiences and is undertaking a multi-faceted approach to reaching diverse individuals.  LSST EPO is planning to partner with at least five organizations serving 1) women/girls, 2) individuals from traditionally underrepresented groups in STEM, and 3) individuals in low socioeconomic communities. Representatives from these organizations will be key stakeholders of the EPO program, helping to shape deliverables and a culturally responsive program evaluation. Furthermore, these relationships will allow for co-creation of EPO deliverables to help ensure materials are accessible to, of interest to, and relevant to diverse populations.

LSST EPO is breaking new ground in bringing astronomical data to the public in a timely, engaging, and easy-to-use way. It is not unreasonable to anticipate tens of millions of public users browsing the LSST sky, exploring discoveries as they are broadcast, and monitoring objects of interest. Results of EPO's ongoing evaluation will be made publicly available, allowing us to share lessons learned, insights, and program impacts with the larger science EPO community.
