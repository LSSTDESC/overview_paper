\section{  INTRODUCTION}

Major advances in our understanding of the Universe have historically arisen
from dramatic improvements in our ability to ``see''. We have developed
progressively larger telescopes over the past century, allowing us
to peer further into space, and further back in time. With the development of
advanced instrumentation -- imaging, spectroscopic, and polarimetric -- we
have been able to parse radiation detected from distant sources over the
full electromagnetic spectrum in increasingly subtle ways.
These data have provided the detailed information needed to construct physical
models of planets, stars, galaxies, quasars, and larger structures, and to probe the
new physics of dark matter and dark energy.

Until recently, most astronomical investigations have focused on small samples
of cosmic sources or individual objects. This is because our largest telescope
facilities typically had rather small fields of view, and those with large
fields of view could not detect very faint sources. With all of our existing
telescope facilities, we have still surveyed only a small fraction of the
observable Universe (except when considering the most luminous quasars).

Over the past two decades, however, advances in technology have made it possible to
move beyond the traditional observational paradigm and to undertake large-scale
sky surveys. As vividly demonstrated by surveys such as the Sloan Digital Sky
Survey (SDSS; \cite{York2000}), the Two Micron All Sky Survey (2MASS; \cite{Skrutskie2006}),
the Galaxy Evolution Explorer (GALEX; \cite{Martin2006}),
and Gaia \cite{Gaia2016} to name but a few, sensitive and accurate
multi-color surveys over a large fraction of the sky enable an extremely broad range of
new scientific investigations. These projects, based on a synergy of advances in
telescope construction, detectors, and above all, information technology,
have dramatically impacted nearly all fields of astronomy
-- and several areas of fundamental physics. In addition, the world-wide attention
received by Sky in Google Earth\footnote{http://earth.google.com/sky/}
(\cite{Scranton2007}), the World Wide Telescope\footnote{http://worldwidetelescope.org/home},
and the hundreds of thousands of volunteers
classifying galaxies in the Galaxy Zoo project (\cite{GalaxyZoo})
and its extensions demonstrate that the impact of sky surveys extends
far beyond fundamental science progress and reaches all of society.

Motivated by the evident scientific progress enabled by large sky surveys,
three nationally-endorsed reports by the U.S. National Academy of Sciences\footnote{
  Astronomy and Astrophysics in the New Millennium, NAS 2001;
  Connecting Quarks with the Cosmos: Eleven Science Questions for the New Century, NAS 2003;
  New Frontiers in the Solar System: An Integrated Exploration Strategy, NAS 2003.
}
concluded that a dedicated ground-based wide-field imaging telescope with an effective aperture
of 6--8 meters ``is a high priority for planetary science, astronomy, and physics
over the next decade.'' The Large Synoptic Survey Telescope (LSST) described here is
such a system. Located on Cerro Pachon in northern Chile,
the LSST will be a large, wide-field ground-based telescope
designed to obtain multi-band images over a substantial fraction of the sky
every few nights. The survey will yield contiguous overlapping imaging of over
half the sky in six optical bands, with each sky location visited close to 1000 times over
10 years. The 2010 report ``New Worlds, New Horizons in Astronomy and Astrophysics''
by the NRC Committee for a Decadal Survey of Astronomy and
Astrophysics\footnote{http://www.nap.edu/catalog.php?record\_id=12951}
ranked LSST as its top priority for large ground-based projects, and in May 2014 the National
Science Board approved the project for construction.
As of this writing, the LSST construction phase is nearing completion.
After initial tests with a commissioning camera and then with the main camera, the
ten year sky survey is projected to begin in 2022.

The purpose of this paper is to provide an overall summary of the main LSST
science drivers and how they led to the current system design parameters
(\S~\ref{Sec:refdesign}), to describe anticipated data products (\S~\ref{Sec:dataprod}),
and to provide a few examples of the science programs that LSST will enable
(\S~\ref{Sec:science}). The community involvement is discussed in \S~\ref{Sec:community},
and broad educational and societal impacts in \S~\ref{Sec:impact}. Concluding
remarks are presented in \S~\ref{Sec:conclusions}. This publication will be maintained
at the arXiv.org site\footnote{http://arxiv.org/abs/0805.2366}, and will also
be available from the LSST website (www.lsst.org). The latest arXiv version of this paper
should be consulted and referenced for the most up-to-date information about the
LSST system.
