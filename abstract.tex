
\begin{abstract}
Major advances in our understanding of the
Universe frequently arise from dramatic improvements in our ability to accurately
measure astronomical quantities. Aided by rapid progress in information
technology, current sky surveys are changing the way we view and study the
Universe. Next-generation surveys will maintain this revolutionary
progress. We describe here the most ambitious survey currently planned in
the optical, the Large Synoptic Survey Telescope (LSST). A vast array
of science will be enabled by a single wide-deep-fast sky survey, and
LSST will have unique survey capability in the faint time domain. The LSST design is
driven by four main science themes: probing dark energy and dark matter,
taking an inventory of the Solar System, exploring the transient optical sky,
and mapping the Milky Way. LSST will be a large, wide-field ground-based system
designed to obtain repeated images covering the sky visible from
Cerro Pach\'{o}n in northern Chile. The telescope will have an 8.4 m
(6.5 m effective) primary mirror, a 9.6 deg$^2$ field of view, and a 3.2 Gigapixel
camera.  The standard observing sequence will consist of pairs of
15-second exposures in a given field, with two such visits in each
pointing in a given night to identify and constrain the orbits of
asteroids.  With these repeats, the LSST system is capable of imaging
about 10,000 square degrees of sky in a
single filter in three clear nights.   The typical 5$\sigma$
point-source depth in a single visit in $r$ will be $\sim 24.5$ (AB).
The system is designed to yield high image quality as well as superb astrometric
and photometric accuracy. The project is in the construction phase and will begin
regular survey operations by 2022. The survey area will
be contained within 30,000 deg$^2$ with $\delta<+34.5^\circ$, and will be imaged multiple times
in six bands, $ugrizy$, covering the wavelength range 320--1050 nm.
About 90\% of the observing time will be devoted to a deep-wide-fast survey mode
which will \B{uniformly} observe a 18,000 deg$^2$
region about 800 times (summed over all six bands) during the anticipated 10 years
of operations, and will yield a coadded map to $r\sim27.5$. These data will result in
databases including 20 billion galaxies and a similar number of stars, and will
serve the majority of the primary science programs. The remaining 10\% of the observing time
will be allocated to special projects such as a Very Deep and Fast time domain
survey, whose details are currently under discussion. We illustrate
how the LSST science drivers led to these choices of system
parameters, and describe the expected data products and their characteristics.
The goal is to make LSST data products  including a relational database of about 32
trillion observations of 40 billion objects available to the public and scientists around the
world -- everyone will be able to view and study a high-definition color movie of
the deep Universe.
\end{abstract}

\keywords{
  astronomical data bases: atlases, catalogs, surveys ---
  Solar System ---
  stars ---
  the Galaxy ---
  galaxies ---
  cosmology
}





% latex lsst; dvips -Ppdf -o lsst.ps lsst.dvi; ps2pdf14 lsst.ps lsst.pdf
